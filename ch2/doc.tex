\section{Дискреные системы с запаздыванием}
Иногда удобно рассматривать $N_{t+1}=f(N_t,..,N_{t-\tau})$. Эквивалентная запись в виде многомерной системы:
$$
u_1(t)=N_t,u_2(t)=N_{t-1},\dots,u_{\tau+1}(t)=N_{t-\tau}\:\Rightarrow\:
u_1(t+1)=f(u_1(t),\dots,u_{\tau+1}(t)),
$$
$$
u_2(t+1)=u_1(t),\ u_3(t+1)=u_2(t),\dots,u_{\tau+1}(t+1)=u_{\tau}(t).
$$
\begin{definition}
Для многомерной системы неподвижной точкой называется $u^*=(u_1^*,\dots,u_n^*)$, такая что 
$$\begin{cases}
u_1^* = f(u_1^*,\dots,u_n^*),\\
u_2^*=u_1^*,\\
\ \vdots\\
u_{\tau+1}^*=u_{\tau}^*.
\end{cases}$$
Для системы с запаздыванием: $ u^*=f(u^*,\dots,u^*).$
\end{definition}
Исследуем на устойчивость: пусть $f(N_t,...,N_{t-\tau})\in C(\mathbb{R}^{\tau})$. Дадим всем точкам приращение:
$$
\begin{array}{c}
u_1(t+1)=u^*+\Delta\,u_1(t+1)\\
u_2(t+1)=u^*+\Delta\,u_2(t+1)\\
\vdots\\
u_{\tau}(t+1)=u^*+\Delta\,u_{\tau}(t+1)\\
\end{array}
$$ 
$$
u^*+\Delta\,u_1(t+1)=f\left(u^*+\Delta\,u_1(t),\dots,u^*+\Delta\,u_{\tau+1}(t)\right)=f(u^*,\dots,u^*)+\sum\limits_{i=1}^{\tau+1}\dfrac{\partial f}{\partial u_i}(u^*,\dots,u^*)\Delta\,u_i(t)+o(|\Delta\,u|)
$$
В первом приближении это линейная система $u(t+1)=Au(t)$.
\begin{lemma}
Собственное значение $A$ удовлетворяет $\mu^{\tau+1}=\sum\limits_{i=1}^{\tau}a_i\mu^{\tau-i}$.
\end{lemma}
\begin{proof}
По индукции

\underline{База:} $n=2$ 
$$
A=\begin{pmatrix}
a_1 & a_2\\
1 & 0
\end{pmatrix},\qquad 
\chi(\mu)=\begin{pmatrix}
a_1-\mu & a_2\\
1 & -\mu
\end{pmatrix}
\Rightarrow \mu^2=\mu a_1+a_2
$$ 

\underline{Переход:} $n=n-1$
$$
\left|
\begin{matrix}
a_0-\mu & a_1  & \dots & a_{\tau-1} & a_{\tau} \\
1 		& -\mu & \dots &     0		& 0	  \\
 \hdotsfor{5} \\
0		& 0	   & \dots &  1	        & -\mu
\end{matrix}
\right| = -\mu\left(\mu^{\tau}-\sum\limits_{i=0}^{\tau-1}a_i\mu_i^{\tau-1-i}\right)+a_{\tau}
$$
\end{proof}
Из леммы выше следует:
\begin{enumerate}
\item Характеристический многочлен $\mu^{\tau+1}=\sum\limits_{i=1}^{\tau}a_i\mu_i^{\tau-i}$.
\item Если все $|\mu_i|<1,\: i=\overline{1,\tau+1}$, то неподвижная точка устойчива 
\item Если существует $|\mu_i|>1$, то неподвижная точка неустойчива. 
\end{enumerate}

\underline{Общий случай:} $u_{t+1}=f(u_t),\ y_t=\left(u_1(t),\dots,u_n(t)\right)^T,\ f=(f_1,\dots,f_n),\ f:\mathbb{R}^n\rightarrow\mathbb{R}^n$.

Матрица Якоби: 
$$
A=\left\|\dfrac{\partial f_i}{\partial u_k}\right\|_{u=u^*}=
\begin{pmatrix}
\dfrac{\partial f_1}{\partial u_1}(u^*) &\dots &\dfrac{\partial f_1}{\partial u_n}(u^*) \\
\vdots &\ddots & \vdots \\
\dfrac{\partial f_n}{\partial u_1}(u^*)&\dots&\dfrac{\partial f_n}{\partial u_n}(u^*)
\end{pmatrix}
$$
\subsection{Падение и взлет численности математической модели популяции жука (tribolium)}
У жуков три стадии развития. Обозначим фазы следующим образом:
$$
L\textbf{--личинка}\longrightarrow P\textbf{--куколка}\longrightarrow A\textbf{--взрослая особь}
$$
Запишем математическую интерпретацию каждой фазы:
$$
\begin{cases}
L_{t+1}=bA_t,  \\
P_{t+1}=(1-\eta_l)L_t,\quad &\text{$\eta_l$--смертность L}\\
A_{t+1}=(1-\eta_p)P_t + (1-\eta_A)A_t,\quad &\text{$\eta_p,\ \eta_A$--смертность P и A}. 
\end{cases}
$$
$$
A=
\begin{pmatrix}
0 & 0 & b \\
1-\eta_l & 0 & 0 \\
0 & 1-\eta_p & 1-\eta_A 
\end{pmatrix}
$$
\begin{multline*}
\left|A-\mu\,E\right| = 
\left|
\begin{matrix}
-\mu & 0 & b \\
1-\eta_l & -\mu & 0 \\
0 & 1-\eta_p & 1-\eta_A -\mu
\end{matrix}
\right| =
\mu^2(1-\eta_A-\mu)+b(1-\eta_p)(1-\eta_l) =\\
= \mu^3 - \mu^2\underbrace{(1-\mu-\eta_A)}_{<1} - b\underbrace{(1-\eta_p)}_{<1}\underbrace{(1-\eta_l)}_{<1}
\end{multline*}
Поскольку $b > 1$, то неподвижная точка $(0,0,0)$ неустойчива.

С учетом каннибализма модель можно  описать так:
$$
\begin{cases}
L_{t+1}=bA_te^{-C_{ea}A_t-C_{al}L_t},  \\
P_{t+1}=(1-\eta_l)L_t,\\
A_{t+1}=(1-\eta_p)P_t e^{-C_{pa}A_t} + (1-\eta_A)A_t. 
\end{cases}
$$
Причем $C_{ea}=0.009,C_{al}=0.012,C_{pa}=0.004,\eta_l=0.267,\eta_p=0,\eta_A=0.0036,b=7.48$
\section{Непрерывные системы}
$
\begin{cases}
\dfrac{dx_i}{dt}=f_i(x),\quad i=\overline{1,n},\quad x=(x_1,x_2,...,x_n)\in\Omega\subset\mathbb{R}^n\\
x_i(0)=x_i^0
\end{cases}$ --- автономная система.\\

Будем называть $\Omega$  фазовым пространством, а множество $\{x(t,x^0)\}$ -- фазовой кривой, и также будем считать, что $f\in C^{\infty}(\Omega)$.  
\subsection{Свойства автономных систем}
Запишем еще раз вид автономной системы:
\begin{equation}\label{ch2.avs}
\begin{cases}
\dfrac{dx}{dt}==f(x),\quad x\in D\subset\mathbb{R}^n,\\
x(0)=x^0.
\end{cases}
\end{equation}
\begin{enumerate}
\item \textit{Если $x=\psi(t)$ удовлетворяет \ref{ch2.avs}, $\psi$ -- гладкая, то $\overline{x}=\psi(t)+C$ --- тоже решение $\forall C=\const$}. 
\begin{proof}
$\dfrac{d\overline{x}}{dt}=\dfrac{d\psi(t+C)}{dt}=\dfrac{\psi(t+C)}{d(t+C)}=f(\psi(t+C))$, т.к. $\dfrac{d\psi}{dt}=f(\psi(t))$
\end{proof}
\item  \textit{Фазовые траектории либо не имеют общих точек, либо совпадают.}
\begin{proof}
Пусть $x^0\in\mathbb{R}^n$ -- точка, через которую проходят две фазовые кривые $\Rightarrow \exists x_1=\varphi(t;x^0),\ x_2=\psi(t;x^0)$ -- решения системы: $\exists\: t_1,t_2:\varphi(t;x^0)=\psi(t;x^0)$.

Проектируем $x_1,x_2$ на фазовую плоскость так, чтобы они пересеклись в $x^0$. Пусть теперь $\chi(t;x^0)=\varphi(t+(t_1-t_2);x^0)$, тогда $\chi$ является решением по первому свойству.

$\chi(t_2;x^0)=\varphi(t_1;x^0)=\psi(t_2;x^0)\Rightarrow$ в точке $t_2$ 
$\chi$ и $\psi$ проходят через $x^0\Rightarrow$ по теореме о единственности кривые совпадают, значит $\varphi=\psi$.  
\end{proof}
\item \textit{Точка $a\in D\subset\mathbb{R}^n$ называется неподвижной точкой, если $f_i(a)=0,i=1,2,...,n \Rightarrow$ неподвижная точка является фазовой траекторией:}
$$
x_i=a_i,\quad \dfrac{dx_i}{dt}=0,\quad f_i(a)=0.
$$
\item \textit{Всякая фазовая траеткория автономной динамической системы может принадлежать одному из трех типов кривых: гладкая кривая без самопересечений, цикл, точка.}
\item \textit{Если фазовая траектория -- цикл, то отвечающее ей решение периодическая функция.}
\begin{proof}
Пусть $\dfrac{dx_i}{dt}=f_i(x),\ i=\overline{1,n},\ x_i(0)=a_i$. Тогда $x=\psi(t),\ a=\psi(0)$. 
$$
dS=\sqrt(dx_1^2+...+dx_n^2) \Rightarrow \dfrac{dS}{dt}=\sqrt{\sum\limits_{i=1}^{n}\dfrac{dx_i}{dt}^2}=\sqrt{\sum\limits_{i=1}^{n}f_i^2}=|f(\psi(t))|
$$
Получили, что $dS=|f(\psi(t)|dt$, длина:
$$
l(t)=\int\limits_{0}^{t}ds=\int\limits_{0}^{t}|f(\psi(t))|dt\geqslant\,mt.
$$ 
Неподвижная точка $(|f(\psi(t))|=0)\notin\sum$, т.к. это противоречило бы свойству 2, то есть $|f(\psi(t))| \neq 0, x\in\sum;\ m\leqslant|f(\psi(t))|\leqslant M,\ m > 0$. 

 
Пусть $L$ -- длина замкнутой фазовой траектории. Имеем: $l(t)$  возрастает, $l(t)\leqslant\,L$. Тогда 
$$
\exists\,T:l(T)=L\Rightarrow L = \int\limits_{0}^{T}|f(\psi(t))|dt
$$
\end{proof}

Рассмотрим следующий пример:

$
\begin{cases}
\dot{x_1} = x_2\\
\dot{x_2} = -x_1
\end{cases}\Rightarrow
\dfrac{dx_1}{dx_2}=-\dfrac{x_2}{x_1}\Rightarrow
x_1^2+x_2^2 = R^2.
$

$\int\limits_{0}^{T}\sqrt{x_1^2+x_2^2}dt=\int\limits_{0}^{T}Rdt=RT\Rightarrow\,2\pi R=RT\Rightarrow T=2\pi.$
\item \textit{Групповое свойство}

$
\begin{cases}
\dfrac{dx}{dt}=f(x),\quad x\in D\subset\mathbb{R}^n,\\
x(0)=x^0.
\end{cases}
\qquad x=x(t;x^0)$  -- решение 

Пусть $t_1,t_2>0\Rightarrow\,x(t_1+t_2;x_0)=x(t_1;x(t_2,x^0))=x(t_2;x(t_1,x^0))$.
\begin{proof} Рассмотрим
$$
\begin{array}{rcl}
\varphi_1(t)&=x(t;x(t_1;x^0)) \qquad \varphi_1(0)&=x(0;x(t_1;x^0)) = x(t_1;x^0), \\
\varphi_2(t)&=x(t+t_1;x^0) 	 \qquad \varphi_2(0)&= x(t_1;x^0). 
\end{array}
$$
Получаем, что решения совпадают при $t=0\Rightarrow$ совпадут при $t=t_2$, в силу единственности
$$x(t_2;x(t_1,x^0))=x(t_1+t_2;x^0)$$ 
\end{proof}
\item \textit{Гладкая невырожденная замена переменных}

$\dfrac{dx_i}{dt}=f_i(x_1,...,x_n), x\in D\subset\mathbb{R}^n,\ x_i=\psi_i(y_1,...,y_n)$

$
\dfrac{dx_i}{dt}=\sum\limits_{k=1}^{n}\dfrac{\partial \psi_i}{\partial y_k}\dfrac{dy_k}{dt}=f(\psi(y_1,...,y_n))
$
$$
J=
\begin{pmatrix}
\dfrac{\partial \psi_1}{\partial y_1}\ldots\dfrac{\partial \psi_1}{\partial y_n}\\
\ddots\\
\dfrac{\partial \psi_n}{\partial y_1}\ldots\dfrac{\partial \psi_n}{\partial y_n}
\end{pmatrix}
$$
Пусть $\exists\left(\dfrac{\partial x}{\partial y}\right)^{-1}\Rightarrow
\dfrac{dx}{dt}=\left(\dfrac{\partial x}{\partial y}\right)\dfrac{dy}{dt}=f(\psi(y)),\quad \dfrac{dy}{dt} = \left(\dfrac{\partial x}{\partial y}\right)^{-1}f(\psi(y))
$
\end{enumerate}
\begin{theorem}[Теорема о выпрямлении векторного поля]
Пусть $a$ не особая точка динамической системы
$$
\dfrac{dx_i}{dt}=f_i(t),\ i=1,2,...,n
$$
Тогда в достаточно малой окрестности точки $a$, сущесвует невырожденная замена переменных $x_i=\psi_i(y_1,...,y_n),\ i=1,2,...,n$, которая переводит систему в систему вида:
$$
\dfrac{dy_1}{dt}=0,\quad \dfrac{dy_2}{dt}=0,\ \ldots\ ,\dfrac{dy_{n-1}}{dt}=0,\quad \dfrac{dy_n}{dt}=1.
$$
\end{theorem} 
\begin{proof}
Не ограничивая общности, будет полагать $|f_n(a)|\neq\,0$, т.е. $f_n(a_1,\ldots,a_n)$ . Рассмотрим задачу Коши:
$$
\begin{cases}
\dfrac{dx_i}{dt}=f_i(x_1,\ldots,x,_n)\\
x_1(0)=\xi_1,\ldots,x_{n-1}(0)=\xi_{n-1}, x_n(0)=a_n
\end{cases}
$$
Обозначим за $a'$ и $\xi'$
$$
a'=(a_1,\ldots,a_{n-1}),\quad \xi'=(\xi_1,\ldots,\xi_{n-1})\in U_{a'}
$$
Для всех $\xi'$ сопоставим соответствующую траекторию, т.е. $\xi'\longleftrightarrow \psi(t;\xi')=x(t)$. Пусть $y_1=\xi_1,\ldots,y_{n-1}=\xi_{n-1},y_n=t$ и сделаем следующую замену: $x(t)=\psi(y_n,y'),\ y'=(y_1,\dots,y_{n-1})$. 

Проверим, что такая замена переменных невырожденная. В силу задачи Коши имеем:
$$
\begin{array}{ll}
\psi_i(0;\xi')=\psi_i(0,y')=\xi_i,\quad i=1,\dots,n-1\\
\psi_n(0;\xi')=\psi(0,y')=a_n
\end{array}
$$
$$
\begin{array}{ll}
\dfrac{\partial\psi_i}{\partial y_j}=\dfrac{\partial\psi(0;\xi')}{\partial\xi_j}=\delta_{ij}=\begin{cases} 1,i=j\\0,i\neq\,j\end{cases}\\
\dfrac{\partial\psi_n}{\partial y_k}=0,\quad \dfrac{\partial \psi_n}{\partial y_n}=\dfrac{d\psi_n}{dt}=f_n(a)\neq 0
\end{array}
$$

$$
J=
\begin{pmatrix}
1 & 0 & \dots & 0 & * \\
0 & 1 &  \dots &  0 & * \\
\vdots &\vdots & \ddots & \vdots& \vdots\\
0  & 0 &\dots  & 1 & *\\
0 & 0 & \dots & 0 & f_n(a)
\end{pmatrix}\Rightarrow \exists\ J^{-1}ю
$$
\end{proof}

Таким, образом структура фазовой траектории в окрестности любой точки, не являющейся положением равновесия, тривиальна. При подходящем выборе координат фазовая траектория представляет собой пучок параллельных прямых.
\subsection{Теорема Лиувилля о скорости изменения фазового объема}
\begin{equation}\label{ch2.eq1}
\begin{array}{rcl}
\dfrac{dx_i}{dt}=f(x),\quad &x\in\Omega\subset\mathbb{R}^n\quad &D_t=\{x=x(t;y)\ |\ y\in D_0\} \\
x(0)=y,\quad &y\in D_0\subset\Omega\quad &V_0=|D_0|,\ V_t=|D_t|
\end{array}
\end{equation}
Рассмотрим связь $x=x(t;y)$. Введем матрицу
$$
\left(\dfrac{\partial x}{\partial y}\right) = 
\begin{pmatrix}
\dfrac{\partial x_1(t;y)}{\partial y_1} & \dots & \dfrac{\partial x_1(t;y)}{\partial y_n}\\
& \ddots & \\
\dfrac{\partial x_n(t;y)}{\partial y_1} & \dots & \dfrac{\partial x_n(t;y)}{\partial y_n}
\end{pmatrix}
$$
где $\dfrac{\partial x_i(t;y)}{\partial y_j}$ -- матрица чувствительности компонент к изменению начальных данных. Введем также матрицу Якоби
$$
\left(\dfrac{\partial f}{\partial x}\right) = 
\begin{pmatrix}
\dfrac{\partial f_1}{\partial x_1} & \dots & \dfrac{\partial f_1}{\partial x_n}\\
& \ddots & \\
\dfrac{\partial f_n}{\partial x_1} & \dots & \dfrac{\partial f_n}{\partial x_n}
\end{pmatrix}
$$
\begin{lemma}[Уравнение в вариацях]\label{ch2.lem1}
Пусть задано (\ref{ch2.eq1}), тогда справедливо следующее матричное уравнение
\begin{equation}\label{ch2.eq2}
\dfrac{d}{dt}\left(\dfrac{\partial x(t;y)}{\partial y}\right) = \left(\dfrac{\partial f(x)}{\partial x}\right)\left(\dfrac{\partial x(t;y)}{\partial y}\right)
\end{equation}
\end{lemma}
\begin{proof}
$$
\dfrac{d}{dt}\dfrac{\partial x_k(t;y)}{\partial y_i}=\dfrac{\partial}{\partial y_i}\dfrac{d}{dt}\left(x_k(t;y)\right)= \dfrac{\partial}{\partial y_i}f_k(x(t;y)) = \sum\limits_{s=1}^{n}\dfrac{\partial f_k}{\partial x_s}\dfrac{\partial x_s(t;y)}{\partial y_i}.
$$
Полученное выражение и есть покомпонентная запись (\ref{ch2.eq2}).
\end{proof}
\begin{lemma}[Лиувилля о дифференцировании определителя]\label{ch2.lem2}
Пусть имеется матрица $A(t)=\|a_{ij}(t)\|$, тогда 
\begin{equation}\label{ch2.eq3}
\dfrac{d}{dt}|A(t)|=|A(t)|\tr\left(\dfrac{dA(t)}{dt}A^{-1}(t)\right)
\end{equation}
\end{lemma}
\begin{proof}

Пусть $a_{ij}(t+\Delta\,t)=a_{ij}(t)+a'_{ij}\Delta\,t + o(\Delta\,t)$,
$\dfrac{dA}{dt}=\|a'_{ij}(t)\|,\ i,j=\overline{1,n}$

Обозначим $B=(\dfrac{dA}{dt}A^{-1})=b_{ij},\:i,j=\overline{1,n}$
$$
|E+B\Delta\,t|=
\left|
\begin{matrix}
1+b_{11}\Delta\,t & \dots & b_{1n}\Delta\,t\\
\vdots & \ddots &\vdots\\
b_{n1}\Delta\,t & \dots & 1+b_{nn}\Delta\,t
\end{matrix}
\right|
$$
Тогда запишем
\begin{multline*}
|A(t+\Delta\,t)=|A(t)+\dfrac{dA}{dt}\Delta\,t + o(\Delta\,t)|=|E+\Delta\,t\dfrac{dA}{dt}A^{-1}(t)+o(\Delta\,t)A(t)|=\\
=|E+\Delta\,t\underbrace{(\dfrac{dA}{dt}A^{-1}(t)}_{B(t)}+o(\Delta\,t)||A(t)| = (1+\Delta\,t\underbrace{(b_{11}+\dots+b_{nn})}_{\tr\:B}+o(\Delta\,t))|A(t)|
\end{multline*}
Получаем
$$
\dfrac{|A(t+\Delta\,t)|-|A(t)|}{\Delta\,t} = \dfrac{|A(t)|\:\tr\left(\dfrac{dA}{dt}A^{-1}(t)\right)\Delta\,t+o(\Delta\,t)}{\Delta\,t}.
$$
Полученная дробь сходится к (\ref{ch2.eq3}).
\end{proof}
\begin{theorem}[Лиувилля]
Пусть задано (\ref{ch2.eq1}) $\Rightarrow\ \dfrac{dV_t}{dt}=\displaystyle\int\limits_{D_t}^{}divf(x)dx_t$.

$div(\underbrace{f_1(x),\dots,f_n(x)}_{\textit{вектор}})=\underbrace{\sum\limits_{i=1}^{n}\dfrac{\partial f_i(x)}{\partial x_i}}_{\textit{скаляр}}$
\end{theorem}
\begin{proof}
$$
dx_t=dx_1(t;y)\dots\,dx_n(t;y)=\{x_1(t;y)\dots\,x(t;y)\rightarrow y_1\dots\,y_n\}=\left|\dfrac{\partial x(t;y)}{\partial y}\right|dy_1\dots\,dy_n
$$
Выражение для $V_t$ перепишется следующим образом:
$$
V_t=\displaystyle\int\limits_{D_t}^{}dx_1(t;y)\dots\,dx_n(t;y)=\displaystyle\int\limits_{D_0}^{}\left|\dfrac{\partial x(t;y)}{\partial y}\right|dy_1\dots\,dy_n\
$$

\begin{multline*}
\dfrac{dV_t}{dt}=\displaystyle\int\limits_{D_0}^{}\dfrac{d}{dt}\left|\dfrac{\partial x(t;y)}{\partial y}\right|dy_1\dots\,dy_n=\{\textit{лемма \ref{ch2.lem2}}\}=
\displaystyle\int\limits_{D_0}^{}\tr
\left(
\dfrac{d}{dt}
\left(\dfrac{\partial x(t;y)}{\partial y}\right)
\left(\dfrac{\partial x(t;y)}{\partial y}\right)^{-1}
\right)\cdot\\
\cdot\left(\dfrac{\partial x(t;y)}{\partial y}\right)dy=
\{\textit{лемма \ref{ch2.lem1}}\}=
\displaystyle\int\limits_{D_0}^{}\tr
\left(
\dfrac{\partial f}{\partial x}
\left(\dfrac{\partial x(t;y)}{\partial y}\right)
\left(\dfrac{\partial x(t;y)}{\partial y}\right)^{-1}
\right)
\left(\dfrac{\partial x(t;y)}{\partial y}\right)dy=\\
=\displaystyle\int\limits_{D_t}^{}\tr\left(\dfrac{\partial f}{\partial x}\right)dx_t=
\displaystyle\int\limits_{D_t}^{}\sum\limits_{i=1}^{n}\left(\dfrac{\partial f_i}{\partial x_i}\right)dx_t=
\displaystyle\int\limits_{D_t}^{}div f(x)dx_t
\end{multline*}
\end{proof}