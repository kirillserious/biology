\section{Система хищник-жертва Гаузе}
Рассмотрим систему, состоящую из популяции хищников и жертв:
\begin{equation}
	\begin{array}{lr}
		\left\lbrace
			\begin{array}{ll}
				\dot{u} &= ug(u) - vp(u)\\
				\dot{v} &= v(-d+q(u))
			\end{array}
		\right. &, u, v \in \mathbb{R}_{+}^{2}.
	\end{array}
	\label{hunted_sys}
\end{equation} 
Где $d > 0$ --- параметр, характеризующий скорость вымирания хищников в отсутствии жертв; $g(u), p(u), q(u)$ --- гладкие функции, на которые дополнительно наложены ограничения: $g(u) > 0$, если $0 < u < K$, $g(K) = 0$, $g(u) < 0$, если $u>K$; $p(0) = 0, p(u) > 0$, если $u > 0; q(0) = 0, q(u) > 0, \dot{q} > 0,$ если $u > 0$.
В системе \ref{hunted_sys} функция $g(u)$ описывает относительный коэффициент скорости роста популяции жертв в отсутствии хищников, если $v=0$, то устойчивым положением равновесия является точка $u = K$. Типичным примером функции $g(u)$ является логистическое уравнение: $g(u) = ru(1-\frac{u}{K})$ или, например $g(u) = (K - u)$. Функция $p(u)$ --- трофическая фугкция хищника. Трофическая функция обычно ограничена сверху, так как вне зависимости от количества жертв один хищник не сможет убить больше определённого количества жертв, типичный пример трофической функци --- $\frac{Au^2}{Bu^2+Du}$, $Bu^2+Du > 0$. Функция $g(u)$ описывает эффективность потребления жертв хищниками и подчиняется тем же ограничениям, что и трофическая функция.
Неподвижными точками системы \ref{hunted_sys} будут точки $O(0,0), A(K,0), B(u^*,v^*)$, где $u^* : q(u^*) = d, v^* = u^* \frac{g(u^*)}{p(u^*)}$. Точка $B$ существует если прямая $u = u^*$ пересекает кривую $u\frac{g(u)}{p(u)}, u^* < K$.
Вычислим матрицу Якоби системы \ref{hunted_sys}:
\begin{equation*}
	\mathcal{J}(u,v) = \left(
		\begin{array}{cc}
			g(u) + ug'(u)-vp'(u) & -p(u)\\
			vq'(u) & -d + q(u)
		\end{array}
	\right).
\end{equation*}
Точка $O$ --- седло, так как 
\begin{equation*}
	\mathcal{J}(O) = 
	\left(
		\begin{array}{cc}
			g(0) & 0\\
			0 & -d + q(0)
		\end{array}
	\right).
\end{equation*}
Прямая $u = 0$ будет устойчивой, а  $v = 0$ --- неустойчивой.
Положение $А$ может быть как устойчиввым, так и неустойчивым: 
\begin{equation*}
\mathcal{J}(A) = 
\left(
\begin{array}{cc}
Kg'(K) & -p(K)\\
0 & -d + q(K)
\end{array}
\right).
\end{equation*}
В случае отрицательности $g'(K)$ усточивость точки $A$ определяется знаком выражения $-d + q(K)$. Если $K > u^*$, то $-d + q(K) > 0$, и точка $A$ --- седло, устойчивое многообразие которого расположенно на оси $v = 0$, а неустойчивое расположено перпендикулярно к устойчивому. В противном случае точка $A$ --- устойчивый узел, что соответствует вымиранию хищников и равновесному существованию жертв в точке $u=K$. 
Если $A$ --- седло, то в $\mathbb{R}_+^2$ существует положение равновесия $B$, которое может быть как устойчивым таак и неустойчивым. Рассмотри матрицу Якоби в этой точке:
\begin{equation*}
\mathcal{J}(B) = 
\left(
\begin{array}{cc}
g(u^*) + u^* g'(u^*) - v^* p'(u^*)& -p(u^*)\\
v^* q'(u^*) & 0
\end{array}
\right) = 
\left(
\begin{array}{cc}
	\mu(u^*) & -p(u^*)\\
	v6* q'(u^*) & 0
\end{array}
\right).
\end{equation*}
Где $\mu(u^*) = p(u^*)\left| \left(\frac{ug(u)}{p(u)} \right)^{'} \right|_{u=u^*}$. Для того, чтобы сделать вывод об устойчивости положения равновесия $B$, выпишем след и определитель матрицы Якоби:
\begin{equation*}
\begin{array}{lr}
\tr{\mathcal{J}(B)} = \mu(u^*), &  \det{\mathcal{J}(B)} = v^* q'(u^*)p(u^*) > 0.
\end{array}
\end{equation*} 
Знак $\mu(u^*)$ определяется тангенсом угла наклона касательной к кривой, заданной уравнением $v = \frac{ug(u)}{px(u)}$/ В случае отрицательности эой величина $B$ --- устойчивый ущел или фокус. Если тангенс угла касательной положителе, то точка $B$ --- неустойчивый узел или фокус. Последнее зависит от знака выражения $\mu(u^*)^2 - 4v^* q'(u^*)p(u^*)$.

\section{Можель хищник-жертва Холлинга}
\begin{equation}
	\begin{array}{ll}
		\dot{N} = rN\left(1 - \frac{N}{k}\right) - \frac{cNP}{A+N}\\
		\dot{P} = p\left(-d + \frac{bN}{A+N}\right),
	\end{array}
\end{equation}
 Где переменные $N, P$ --- численности жертв и хищников соответственно, а остальные параметры неотрицательны. При этом вышеуказанная модель является моделью Гаузе с параметрами $g(N) = 1 - \frac{N}{K}, p(N) = \frac{CN}{A+N}, q(N)=\frac{BN}{A+N}$. 
 Исследуем нетривиальное положение равновесия системы $B = (N^*,P^*)$, где $P^* - \left(K - N^*\right) \left( A + N^* \right) / \left( CK \right)$. Так как $P_{N}^{'}(N) = \left( -2N+K-A \right) / \left( KC \right)$, то условие существования замкнутой траектории приобретает вид:
 
 \begin{equation*}
	 N^* < \frac{K+A}{2}.
 \end{equation*}
 При  $N^* > \frac{K+A}{2}$, $B$ асимптотически устойчиво 
