\documentclass{article}
\usepackage[utf8]{inputenc}

\usepackage[russianb]{babel}
\usepackage{vmargin}
\setpapersize{A4}
\setmarginsrb{2.5cm}{2cm}{1.5cm}{2cm}{0pt}{0mm}{0pt}{13mm}
\usepackage{indentfirst}
\usepackage{amsmath}
\sloppy
\begin{document}

\title{ДСиБМ}
\author{arin7al}
\date{December 2019}

Если система \((1)\) имеет ! пол. равн  \(p \in R^n \), тогда \(p\) является единственно представимой треакторией системы \((1)\), \(a_{ii} > 0\), \(i = \overline{1,n}\).
Если \(a_{ii} = 0\), то \(\exists\) ПИ сиситемы \(\nu(\upsilon):\dot{\nu}(\upsilon) = 0\),
траектории системы \(\nu(\upsilon) = const\).

\vspace{0.5cm}
неподв. \(T:\omega_i(p) = 0\), \(i = 1,2,...,n\).

\vspace{0.5cm}
\(
 \begin{cases}
  	a_{11}p_1 + a_{12}p_2 = r_1
  	\\
  	a_{21}p_1 + a_{22}p_2 - a_{23}p_3 = r_2
  	\\
  	...
  	\\
  	a_{i,i-1}p_{i-1} + a_{ii}p_i - a_{i,i+1}p_{i+1} = r_i &\textit{\hspace{10pt}\(r_i\) выражается через \(p\) и подставляется в \((1)\)}
  	\\
  	...
  	\\
  	a_{n,n-1}p_{n-1} - a_{nn}p_n = r_n
 \end{cases}
\)

\vspace{0.5cm}
\(\nu(\upsilon) = \sum_{i=1}^n c_i(\upsilon_i - p_i\ln{u_i})\), где \(c_i\) -- некоторые константы

\vspace{0.5cm}
\(\dot{\nu}(\upsilon) =\sum_{i=1}^n c_i(1-\frac{p_i}{\upsilon_i})\dot{\upsilon_i} = \sum_{i=1}^n c_i(u_i - p_i)\omega_i(\upsilon) = \lbrace \frac{d\upsilon_i}{dt} = \upsilon_i\omega_i(\upsilon)\rbrace \),  обозначим \(\nu_i = \upsilon_i - p_i\)

\vspace{0.5cm}
\(\omega_i = a_{i,i-1}(\upsilon_{i-1}-p_{i-1}) - a_{ii}(\upsilon_i - p_i) - a_{i,i+1}(\upsilon_{i+1} - p_{i+1}) = a_{i,i-1}\nu_{i-1} - a_{ii}\nu_i - a_{i,i+1}\nu_{i+1}\)


\vspace{0.5cm}
\(\dot{\nu_i}(\upsilon) = -\sum_{i=1}^n c_ia_{ii}\nu_i^2 + \sum_{i=1}^{n-1} \nu_i\nu_{i+1}(-c_i a_{i+1} + c_{i+1}a_{i+1,i})\)

\vspace{0.5cm}
\(\frac{c_i}{c_{i+1}} = \frac{a_{i+1,i}}{a_{i,i+1}}\), \(c_i > 0\).

\vspace{0.5cm}
\(\dot{\nu_i}(\upsilon) = -\sum_{i=1}^n c_ia_{ii}\nu_i^2 \leq 0 \)

\vspace{0.5cm}
\(\dot{\nu_i}(\upsilon) = 0\), \hspace{10pt}\({\nu_i}(\upsilon) = -\sum c_i(\upsilon_i - p_i\ln \upsilon_i)\), \hspace{10pt} \({\nu_i}(\upsilon) = const\)

(единств. представ. точка, так бывает редко)

\vspace{0.5cm}
\subsection*{{\Large \underline{Модель циклической конкуренции}}}

\vspace{0.5cm}
\(\dot{\upsilon_1} = \upsilon_1(1-\upsilon_1 - \alpha\upsilon_2 - \beta\upsilon_3) \), \hspace{3cm}\( \upsilon \in R_+^3\) \hspace{2cm} (a) \(0 < \beta < 1 < \alpha\)

\(\dot{\upsilon_2} = \upsilon_2(1-\beta\upsilon_1 - \upsilon_2 - \alpha\upsilon_3) \), \hspace{6.3cm} (в) \(\alpha+\beta\geq 2\) \(b_2\)

\(\dot{\upsilon_3} = \upsilon_3(1-\alpha\upsilon_1 - \beta\upsilon_2 - \upsilon_3) \),
\hspace{6.6cm} \(\alpha+\beta\geq 2\) \(b_1\)

\vspace{0.5cm}
Рассмотрим случай { \underline{\(\upsilon_3 = 0\)}}

\vspace{0.5cm}
({\romannumeral 1})
\(
 \begin{cases}
    \dot{\upsilon_1} = \upsilon_1(1-\upsilon_1 - \alpha\upsilon_2 ) 
    \\
    \dot{\upsilon_2} = \upsilon_2(1-\beta\upsilon_1 - \upsilon_2 ) 
 \end{cases}
\)

\vspace{0.5cm}
\(
J(\upsilon_1,\upsilon_2) = 
 \begin{pmatrix}
 1-2\upsilon_1-2\upsilon_2 & \alpha\upsilon_1\\
 -\beta\upsilon_2 & 1-\beta\upsilon_1 - 2\upsilon_2
 \end{pmatrix}
\)

\vspace{0.5cm}

\(
 J(0,1) = 
 \begin{pmatrix}
 1-\alpha & 0 \\ 
 -\beta & -1
 \end{pmatrix}
\)\hspace{0.5cm}\(\Rightarrow\) это сток.

\vspace{0.5cm}
\(
 J(1,0) = 
 \begin{pmatrix}
 -1 & -\alpha \\ 
 0 & 1 - \beta
 \end{pmatrix}
\)\hspace{0.5cm}\(\Rightarrow\) это седло.

\vspace{0.5cm}
\(
 J(0,0) = 
 \begin{pmatrix}
 1 & 0\\
 0 & 1
 \end{pmatrix}
\)\hspace{0.5cm}\(\Rightarrow\) это исток.

\newpage
%=======================след стр================================================
Рассмотрим случай { \underline{\(\upsilon_2 = 0\)}}

\vspace{0.5cm}
({\romannumeral 2})
\(
 \begin{cases}
 	\dot{\upsilon_1} = \upsilon_1(1-\upsilon_1 - \beta\upsilon_3)
 	\\
 	\dot{\upsilon_3} = \upsilon_3(1-\alpha\upsilon_1 - \upsilon_3)
 \end{cases}
\)

\vspace{0.5cm}
\(\exists!\) траектория, которая обходит точки  A, B и C
%что-то там еще, не понял, что написано
(гомоклинические траектории).

\vspace{0.5cm}
{\underline{Неподвижные точки в трехмерном пространстве}}

\vspace{0.5cm}
\(
 \begin{matrix}
 0(0,0,0)\\
 A(1,0,0)\\
 B(0,1,0)\\
 C(0,0,1)
 \end{matrix}
\)\hspace{2cm}
\(
 \begin{cases}
 	\upsilon_1 + \alpha\upsilon_2 + \beta\upsilon_3 = 1
 	\\
 	\beta\upsilon_1 + \upsilon_2 + \alpha\upsilon_3 = 1
 	\\
 	\alpha\upsilon_1 + \beta\upsilon_2 + \upsilon_3 = 1
 \end{cases}
\)

\vspace{0.5cm}
и мб. \(M ? \in intR_+^3\) средняя неподвижная траектория.

\vspace{0.5cm}
\(
	\Delta = 
	\begin{vmatrix}
	1 & \alpha & \beta \\ 
	\beta & 1 & \alpha \\
	\alpha & \beta & 1
	\end{vmatrix}
	= 1 - \beta^3 + \alpha^3 - 3\alpha\beta + 0
\)

\vspace{0.5cm}
\((1 + \alpha + \beta)(\upsilon_1 + \upsilon_2 + \upsilon_3) = 3 \);\hspace{0.5cm}
\(\upsilon_1 = \upsilon_2 = \upsilon_3 = \frac{1}{1 + \alpha + \beta} = \gamma \)

\vspace{0.5cm}
\(
J (\upsilon_1,\upsilon_2,\upsilon_3) = 
	\begin{pmatrix}
		1 - 2\upsilon_1 - \alpha\upsilon2 - \beta\upsilon-3 & -\alpha\upsilon_1 & -\beta\upsilon_1 \\
		-\beta\upsilon_2 & 1 - \beta\upsilon_1 - 2\upsilon_2 - \alpha\upsilon_3 & -\alpha\upsilon_2\\
		-\alpha\upsilon_3 & -\beta\upsilon_3 & 1 - \alpha\upsilon_1 - \beta\upsilon_2 - 2\upsilon_3
	\end{pmatrix}
\)

\vspace{0.5cm}
\(
J(0,0,0) = 
	\begin{pmatrix}
		1 & 0 & 0\\
		0 & 1 & 0\\
		0 & 0 & 1
	\end{pmatrix}
\)
\hspace{0.2cm} - исток

\vspace{0.5cm}
\(
J(A) = 
	\begin{pmatrix}
	-1 & -\alpha & -\beta \\
	0  & 1-\beta & 0      \\
	0  &   0     & 1-\alpha
	\end{pmatrix}
\)
\hspace{0.2cm} - седло. \(J(B), J(C)\) тоже седло.

\vspace{0.5cm}
\begin{LARGE}
\(
J(M) =
	\begin{pmatrix}
		\frac{-1}{1+\alpha + \beta} & \frac{-\alpha}{1+\alpha+\beta} & \frac{-\beta}{1+\alpha+\beta}\\
		 \frac{-\beta}{1+\alpha+\beta} & \frac{-1}{1+\alpha + \beta} & \frac{-\alpha}{1+\alpha+\beta}\\
		 \frac{-\alpha}{1+\alpha+\beta} & \frac{-\beta}{1+\alpha+\beta} & \frac{-1}{1+\alpha + \beta}\\
	\end{pmatrix}
=\frac{-1}{1+\alpha+\beta}
	\begin{pmatrix}
		1 & \alpha & \beta\\
		\beta & 1 & \alpha\\
		\alpha & \beta & 1
	\end{pmatrix}
\)
\end{LARGE}

\vspace{0.5cm}
{\underline{\begin{large}
Лемма.
\end{large}}}
\hspace{0.1cm}{\large (о собственных значениях и собственных векторах циркуленты)}

\vspace{0.5cm}
\begin{Large}
\(
C = 
	\begin{pmatrix}
	c_0 & c_1 & c_2 & ... & c_{n-1} \\
	c_{n-1} & c_0 & c_1 & ... & c_{n-2} \\
	c_{n-2} & c_{n-1} & c_0 & ... & c_{n-3}\\
	... & ... & ... & ... & ... \\
	c_1 & c_2 & c_3 & ... & c_0
	\end{pmatrix}
\)
\end{Large}
\hspace{0.2cm}
\begin{large}
\(
\rho_k\) - корни, \(\rho^n = 1, \rho_k = e^{\frac{i\alpha\pi k}{n}}, k = 0,1,2,...,n-1.\)

\end{large}

\vspace{0.2cm}

\(
\lambda_k = c_0 + c_1\rho_k + c_2\rho_k^2 + c_3\rho_k^3 + ... +c_{n-1}\rho_k^{n-1}
\);\hspace{0.1cm}\(k = 0,1,2,...,n-1; 
\lambda_k \longrightarrow U^k = (1,\rho_k,\rho_k^2,...,\rho_k^{n-1}).\)
\newpage
%=======================след стр================================================
\begin{center}
\(
 c_0 + c_1\rho_k + c_2\rho_k^2 + c_3\rho_k^3 + ... +c_{n-1}\rho_k^{n-1} = r_k  | *\rho_k
 \),\hspace{0.1cm}\((\rho_n^n = 10\)

\vspace{0.5cm}
\(
\begin{matrix}
	c_{n+1}+ c_0\rho_k + c_1\rho-k^2 + ... + c_{n-2}\rho_k^{n-1} = r_k\rho_k\\
	...\\
	c_1+ c_2\rho_k + c_3\rho-k^2 + ... + c_0\rho_k^{n-1} = r_k\rho_k
\end{matrix}
 \)

\end{center}
\vspace{0.5cm}
\(
C
	\begin{pmatrix}
		1\\ \rho_k \\ \rho_k^2 \\ \vdots \\ \rho_k^{n-1}
	\end{pmatrix}
	=U^k = 
	r_k
	\begin{pmatrix}
		1\\ \rho_k \\ \rho_k^2 \\ \vdots \\ \rho_k^{n-1}
	\end{pmatrix}
\)\hspace{1cm}
\(
	\begin{matrix}
		CU^k = r_kU^k \\
		U^k = (1, \rho_k, \rho_k^2, \cdots, \rho_k^{n-1}) \\
		(c - r_kE)U^k = 0\\ 
		r_k - 
	\end{matrix}
\)собств. знач.

\vspace{0.5cm}
\begin{large}
\(
J(M) 
=\frac{-1}{1+\alpha+\beta}
	\begin{pmatrix}
		1 & \alpha & \beta\\
		\beta & 1 & \alpha\\
		\alpha & \beta & 1
	\end{pmatrix}
\)


\vspace{0.5cm}
\(
	\rho_0 = 1, \rho_1 = \exp^{\frac{2\pi i}{3}} = -\frac{1}{2}+i\frac{\sqrt{3}}{2},
\)\hspace{0.1cm}
\(
	\rho_2 = \exp^\frac{4\pi i}{3} = -\frac{1}{2} - i\frac{\sqrt{3}}{2}
\)

\vspace{0.5cm}
\(
\lambda_0 = (1 + \alpha + \beta)(-\frac{1}{1 + \alpha + \beta}) = -1
\)

\vspace{0.5cm}
\(
\lambda_1=r_1 = (1 + \alpha( -\frac{1}{2}+i\frac{\sqrt{3}}{2}) + \beta( -\frac{1}{2} - i\frac{\sqrt{3}}{2}))(-\frac{1}{1 + \alpha + \beta}) = \frac{1}{1 + \alpha + \beta}(-1+\frac{\alpha+\beta}{2})
\)

\vspace{0.5cm}
\(
\lambda_2 = r_2 = (1 + \alpha( -\frac{1}{2} - i\frac{\sqrt{3}}{2}) + \beta( -\frac{1}{2} + i\frac{\sqrt{3}}{2}))(-\frac{1}{1 + \alpha + \beta})=\frac{1}{1 + \alpha + \beta}(-1+\frac{\alpha+\beta}{2})
\)

\vspace{0.5cm}
\(
U = (1, \exp^\frac{2i\pi}{3}, \exp^\frac{4i\pi}{4})
\)

\vspace{0.5cm}
\textit{в1)}

\(
\alpha + \beta < 2
\)
%$$$$$$$$$$$$$$$$$$$$$$$$$$$$$$$$$$$$$$$место для графиков

\vspace{0.5cm}
\textit{в2)}

\(
\alpha + \beta \geq 2
\)
\hspace{0.1cm} картина неустойчива, решение - \(\alpha + \beta = 2\)

\vspace{0.5cm}
\(
	\begin{matrix}
		S = u_1 + u_2 + u_3\\
		p = u_1u_2u_3
	\end{matrix}
\)\hspace{1cm}
\(
\begin{matrix}
	\dot{S} = S - (u_1^2 + u_2^2 + u_3^2 +(\alpha + \beta)u_1u_2 + (\alpha + \beta)u_1u_3 + (\alpha + \beta)u_2u_3) =\\ 
		= S - S^2
\end{matrix}
\)

\vspace{0.5cm}
\(
\dot{p} = \dot{u_1}u_2u_3 + u_1\dot{u_2}u_3+u_1u_2\dot{u_3}=p(3-u_1(1+\alpha + \beta)=-u_2(1+\alpha+\beta - u_3(1+\alpha + \beta)) = 3p(1-s)
\)

\newpage
\(
	\begin{cases}
	\dot{S} = S(1-S)
	\\
	\dot{p} = 3p(1-s)
	\end{cases}
\)

\vspace{0.5cm}
\(
u_1 + u_2 +u_3 = 1
\) - плоскость

\vspace{0.5cm}
\(
u_1u_2u_3 = 0
\) - (точки, леж. на осях)

\vspace{0.5cm}
\(
V(u) = \frac{p}{S^3}
\) \hspace{0.2cm}
\(
\dot{V} = S^{-1}p \left[ 1 - \frac{(\alpha + \beta)}{2} \right]((u_1 - u_2)^2 + (u_2-u_3)^2 +(u_3-u_1)^2))\leq 0
\)

\vspace{0.5cm}
\(
\alpha + \beta > 2
\)

\vspace{0.5cm}
\(
 p = 0 - 
\) единственная траектория. Предельное мн-во.

\vspace{0.5cm}
\underline{Общие уравнения Лотни-Вольтера}

\vspace{0.5cm}
\(
\dot{u_i} = u_i(r_i-(Au)_i) , 
\)\hspace{0.2cm}
\(
i = 1,2,...,n
\)
\hspace{0.2cm}
\(
u \in R_+^n
\)


\vspace{0.5cm}
\(
A - 
\) ланшафт приспособленности \underline{(1)}


\vspace{0.5cm}
\(
A = 
	\begin{pmatrix}
	a_11 & \ldots & a_{1n} \\
	\vdots & & \vdots \\
	a_{1n} & \ldots & a_{nn}
	\end{pmatrix}
\)\hspace{0.7cm}
\(
(Au)_i = \Sigma_{j=1}^n a_{ij}u_j
\)

\vspace{0.5cm}
\underline{уравнения Колмогорова}

\vspace{0.5cm}
\(
\dot{u_i} = u_if_i(u)
\)

\(
i = 1,2,...,n
\)

\vspace{0.5cm}
Решения не должны уходить на $\infty$ 

$\Rightarrow$ вводитс понятие $1)$ \textit{равномерной ограниченности}:

$\exists S_R : $ траектории $u_0 \in S_R$, \hspace{0.2cm} $u(t;u_0) \in S_R$

2) \textit{перманетности} (усл. невырожденности): если $\forall\delta \exists\varepsilon > 0:$

$u|_{t=0} = u_0 \geq \delta > 0$ \hspace{0.2cm} $\lim_{t \to \infty} u(t) \geq\varepsilon$

\vspace{0.5cm}
$M - $ поглощающая точка, если траектория в нее входит. $u_k=0$

$M -$ неподвижная точка системы (1), когда траектория может в нее попасть

Найти с.з. и с.в. в $M_k$

(с.в. обраует ненулевой угол с плоскоостью.)


$\dot{u_k} = u_k(r_k - (Au)_k)$

возмем $ V = u_k$
тогда $ \dot{V} = \dot{u_k} = u_k(r_k - (Au)_k) > 0$\hspace{0.2cm} в окр-ти нуля будет хуодить

$r_k = (A\overline{u})_k$ \hspace{0.2cm} $\overline{u} = M_k$

если $<0$, то эта точка поглощающая.

\underline{Пример.}

\vspace{0.5cm}
\(
	\begin{cases}
	\dot{u_1} = u_1(1-u_2-\alpha u_1) = u_1 f_1(u_1,u_2)
	\\
	\dot{u_2} = u_2(\gamma - u_1 - \beta u_2) = u_2f_2(u_1,u_2)
	\end{cases}
\)

\newpage
%===================================================================================
$M_1 = \overline{u_1} = (0 , \frac{\gamma}{\beta}),$\hspace{0.2cm}$M_2 = \overline{u_2} = (\frac{1}{\alpha} , 0)$

\vspace{0.5cm}
$
	\begin{matrix}
	M_1 = \overline{u_1} = (0 , \frac{\gamma}{\beta}), & f_1(M_1) = 1 - \frac{\gamma}{\beta} < 0,&\beta < \gamma,  & M_1- {\text {поглощающая т.}}\\
	M_2 = \overline{u_2} = (\frac{1}{\alpha} , 0), & f_2(M_2) = \gamma - \frac{1}{\alpha}<0,&\alpha\gamma < 1, & M_1- {\text {поглощающая т.}}
	\end{matrix}
$


\vspace{0.5cm}
\textbf{\underline{Теорема}}

Пусть задана общая система Лотни-Вольтера:

\underline{(1)} $\dot{u_i} = u_i(r_i - (Au)_i)$ \hspace{0.2cm} $i = 1,2,...,n$ \hspace{0.2cm} $u \in R_+^n$
 
Если $\exists ! $ неподвижная точка системы:

\underline{(2)} $(Ap)_i = r_i, $\hspace{0.2cm}$ i = 1,2,...,n$\hspace{0.2cm}$ p \in intR_+^n$


$\Rightarrow$ система Л-В (1) - невырожденная и равномерно-ограниченная (биологически это

устойчива) и тогда выполняется следущее свойство:

\vspace{0.2cm}
\begin{center}
\begin{Large}
$
\lim_{t \to +\infty} \frac{1}{t}\int_0^t u_0(t)dt = p_i
$
\end{Large}

\end{center}
 
Средний интеграл в системе дает эту неподвижную точку.

\vspace{0.5cm}
Рассмотрим систему (2). Предположим, что эта система не имеет решений.

Тогда 2 случая:

\begin{center}
1) система (2) несовместная

2)система (2) имеет бесчисленное множество решений.
\end{center}

Тогда $\exists \Sigma$ систмы (2) $ dim \Sigma = n - r $,где  $г - $ ранг этой системы.

\vspace{0.5cm}
$u_i = 0$ имеем гиперплоскость системы.

оно пересечет множество, где $u_i = 0$

$
\Sigma\cap(u_i = 0) \neq\Lambda \textit{(не пусто)}
$, но тогда не выполнено условие невырожденности.
Пусть система несовместна $\Rightarrow$ у нее нет решений $>0$.

Рассмотрим множество $L = \lbrace u : r_i - (Au)_i; i = 1,2,...,n\rbrace$
тогда $>0$.

$\Longrightarrow$ обозначим $y_i = r_i - (Au)_i$

$L = \lbrace y : y_i>0\rbrace$

$y $ отделено от нуля, выпукло. По теореме об отделимости $ \exists $ гиперплоскость, 

азделяющая это множество и 0.

\vspace{0.5cm}
Рассмотрим $ c \bot $ этой плоскости $v(y) = (c,y) = \Sigma_{i=1}^n c_i y_i > 0$

$ \dot{v}(u) = \Sigma_{i=1}^n c_i \ln u_i $

$ \dot{v}(u) = \Sigma c_i(r_i - (Au)_i) = \Sigma c_i y_i > 0 $

$\Rightarrow$ вдоль траектории система $\infty$ растет.









\end{large}

\end{document}





\hspace{0.2cm}
\vspace{0.5cm}
\(

\)































